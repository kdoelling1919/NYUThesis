% !TEX root=../main.tex

Writing a dissertation without perception is definitely more difficult. Both in terms of accuracy and reaction times.

\begin{table}[H]
  \centering
  \begin{tabular}{|r|r|r|r|}
  \hline
  RT (s) with each sense & Vision & Audition & Somatosensory \\
  \hline\hline
  Present & 62.3 & 32.9 & 54 \\
  \hline
  Absent & 502 & 249 & 62\\
  \hline
  \end{tabular}
  \caption[Check out the lines on that table!]
  {\textit{Check out the lines on that table!} It's got a bunch of numbers too.}
  \label{tab:examp}
\end{table}

\autoref{tab:examp} is a table. It has no meaning. Don't try to figure it out\footnotemark.
\footnotetext{Its worth noting that nothing in any of this text has any meaning.
Its just gobbledygook to show examples of how to use latex in various settings.}
